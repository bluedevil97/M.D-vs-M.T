
\documentclass[preprint,12pt]{elsarticle}

\usepackage[spanish]{babel}
\usepackage{amssymb}
\usepackage{graphicx}
\usepackage{lineno}
\usepackage[utf8]{inputenc}
\usepackage{url}
\usepackage{color}
\usepackage{enumerate} 
\usepackage[hidelinks]{hyperref}


\begin{document}
	
	\begin{frontmatter}
		
		
		\title{\huge Metodología Inmon vs Metodología Kimball}
		
		\author{Mamani Ayala, Brandon        (2015052715)}
		\author{Quispe Mamani, Angelo	      (2015052826)}
		\author{Vizcarra Llanque, Jhordy	      (2015052719)}
		\author{Ordoñez Quilli, Ronald          (2015052821)}
		\author{Rodriguez Mamani, Juan      (2017057862)}
		
		\address{Tacna, Perú}
		
		\begin{abstract}
			%% Text of abstract
			
The data warehouses in English take each importance day, as organizations move from schemes of only data collection to schemes of analysis of the same. However, in spite of the great diffusion of the concepts related to data warehouses, there is not too much Information available in Spanish regarding the methodologies fo implement them In this short article we will try to provide a general explanation of one of the most used methodologies, the Kimball methodology 
		\end{abstract}
\end{frontmatter}
%%

	
	%%
	%\linenumbers
	
	%% main text
	\section{Resumen}
Los almacenes de datos (data warehouses en inglés) toman cada día mayor importancia, a medida que las organizaciones pasan de esquemas de sólo recolección de datos a esquemas de análisis de los mismos. Sin embargo a pesar de la gran difusión de los conceptos relacionados con los almacenes de datos, no existe demasiada información disponible en castellano en cuanto a las metodologías para implementarlos. En este breve artículo intentaremos brindar una explicación general de una de las metodologías más usadas, la metodología de Kimball \\
	%%
	
	%%
	%\linenumbers
	
	%% main text

\section{Objetivos}

	%%
	
	%%
	%\linenumbers
	
	%% main text

\section{Marco Teorico}



\section{Ejemplo}
 
\section{Ventajas y Desventajas}

\section{Diferencias}

\section{Conclusion}


%%
	
	%%
	%\linenumbers
	
	%% main text

	
	
	\newpage
	
		%ESTILO
	   \bibliography{BIBLIOGRAFIA}		%ARCHIVO .bib
	   \begin{thebibliography}{0}
              \bibitem{Ronald} 
 	    \bibitem{Angelo} 
                 \bibitem{Juan} http://tdan.com/data-warehouse-design-inmon-versus-kimball/20300
                  \bibitem{Jhordy} https://blog.bi-geek.com/arquitectura-comparativa-inmon-y-kimball/
                  \bibitem{Jhordy} https://churriwifi.wordpress.com/2010/04/19/15-2-ampliacion-conceptos-del-modelado-dimensional/
                    \bibitem{Brandon} https://twooctobers.com/blog/8-data-storytelling-concepts-with-examples/
                   \bibitem{Brandon}https://www.ucasal.edu.ar/htm/ingenieria/cuadernos/archivos/5-p56-rivadera-formateado.pdf

         \end{thebibliography}
	
\end{document}

%%
%% End of file `elsarticle-template-1-num.tex'.
